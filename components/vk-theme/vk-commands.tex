% Dynamically sized mid bar.
\newcommand{\bigmid}{\mathrel{\Big|}}

% ---- Colors and Notes ----
\definecolor{dblue}{RGB}{39,135,245}
\definecolor{dlblue}{RGB}{216, 235, 255}
\definecolor{dgreen}{RGB}{124, 155, 127}
\definecolor{dpink}{RGB}{207, 166, 208}
\definecolor{dyellow}{RGB}{255, 248, 199}
\definecolor{dgray}{RGB}{46, 49, 49}
\definecolor{dgray2}{RGB}{200, 200, 200}
\definecolor{ddefblue}{RGB}{226, 239, 254}

% URL
\newcommand{\durl}[1]{\textcolor{dblue}{\underline{\url{#1}}}}

% Circled Numbers
\newcommand*\circled[1]{\tikz[baseline=(char.base)]{\node[shape=circle,draw=dgray2,inner sep=1.2pt,fill=dgray2,line width=1mm] (char) {\normalsize{#1}};}}
% From: http://tex.stackexchange.com/questions/7032/good-way-to-make-textcircled-numbers

% Circled checkmark
\newcommand*\circledmark[1]{\tikz[baseline=(char.base)]{\node[shape=circle,draw,inner sep=1.2pt,line width=0.3mm] (char) {\checkmark};}}

% Under set numbered subset of equation
\newcommand{\numeq}[3]{\underset{\textcolor{#2}{\circled{#1}}}{\textcolor{#2}{#3}}}

% ---- Abbreviations -----
\newcommand{\tc}[2]{\textcolor{#1}{#2}}
\newcommand{\ubr}[1]{\underbrace{#1}}
\newcommand{\uset}[2]{\underset{#1}{#2}}
\newcommand{\eps}{\varepsilon}

% Typical limit:
\newcommand{\nlim}{\underset{n \rightarrow \infty}{\lim}}
\newcommand{\nsum}{\sum_{i = 1}^n}
\newcommand{\nprod}{\prod_{i = 1}^n}

% Add an hrule with some space
\newcommand{\spacerule}{\begin{center}\hdashrule{2cm}{1pt}{1pt}\end{center}}

% Mathcal and Mathbb
\newcommand{\ra}{\rightarrow}
\newcommand{\la}{\leftarrow}

% softmax
\DeclareMathOperator*{\softmax}{softmax}

% ---- Figures, Boxes, Theorems, Etc. ----

% Basic Image
\newcommand{\img}[2]{
\begin{center}
\includegraphics[scale=#2]{#1}
\end{center}}

% Specifications and command for adding arrows to a frame
\tikzset{
    myarrow/.style={
        draw,
        fill=VKblue,
        single arrow,
        minimum height=3.5ex,
        line width=0pt,
        single arrow head extend=0.4ex
    }
}

\newcommand{\arrowdown}{%
\tikz [baseline=-1ex]{\node [myarrow,rotate=-90,  yscale=2.0] {};}
}

% Put a fancy box around things.
\newcommand{\dbox}[1]{
\begin{mdframed}[roundcorner=0pt, backgroundcolor=ddefblue, linewidth=0]
\vspace{1mm}
{#1}
\vspace{1mm}
\end{mdframed}
}

\newcommand{\dblock}[2]{
\begin{tabular}{l}
    \fontsize{30}{40}\selectfont \textbf{#1} \\[2mm]
    \fontsize{10}{14}\selectfont #2
\end{tabular}
}

\newcommand{\dblocksep}[2]{
\begin{tabular}{p{3cm}}
    \fontsize{20}{30}\selectfont \textbf{#1} \\
    \color{VKblue}\rule[1ex]{8ex}{2pt} \\
    \fontsize{8}{12}\selectfont #2
\end{tabular}
}

%  --- PROOFS ---

% Inner environment for Proofs
\newmdenv[
  topline=false,
  bottomline=false,
  rightline = false,
  leftmargin=10pt,
  rightmargin=0pt,
  innertopmargin=0pt,
  innerbottommargin=0pt
]{innerproof}

% Proof Command
%\newenvironment{dproof}{\begin{proof} \text{\vspace{2mm}} \begin{innerproof}}{\end{innerproof}\end{proof}\vspace{4mm}}
\newenvironment{dproof}[1][Proof]{\begin{proof}[#1] \text{\vspace{2mm}} \begin{innerproof}}{\end{innerproof}\end{proof}\vspace{4mm}}


% Dave Definition
\newcounter{DaveDefCounter}
\setcounter{DaveDefCounter}{1}

\newcommand{\ddef}[2]
{
\begin{mdframed}[roundcorner=0pt, backgroundcolor=ddefblue,linewidth=0]
\vspace{1mm}
{\bf #1: {\it #2}}
\stepcounter{DaveDefCounter}
\vspace{1mm}
\end{mdframed}
}

% Block Quote
\newenvironment{dblockquote}[2]{
\begin{blockquote}
#2
\vspace{-2mm}\hspace{10mm}{#1} \\
\end{blockquote}}

% Algorithm
\newenvironment{dalg}[1]
{\begin{algorithm}\caption{#1}\begin{algorithmic}}
{\end{algorithmic}\end{algorithm}}

% Dave Table
\newenvironment{dtable}[1]
{\begin{figure}[h]
\centering
\begin{tabular}{#1}\toprule}
{\bottomrule
\end{tabular}
\end{figure}}

% For numbering the last of an align*
\newcommand\numberthis{\addtocounter{equation}{1}\tag{\theequation}}



\newtheorem{assumption}{Assumption}
\newtheorem{conjecture}{Conjecture}
% \newtheorem{corollary}{Corollary}
\newtheorem{claim}{Claim}
% \newtheorem{example}{Example}
% \newtheorem{lemma}{Lemma}
\newtheorem{proposition}{Proposition}
\newtheorem{remark}{Remark}
% \newtheorem{theorem}{Theorem}
\newtheorem{question}{Question}

\newcommand*{\horzbar}{\rule[.5ex]{2.5ex}{0.5pt}}